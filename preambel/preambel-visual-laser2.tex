%%% Used packages
\usepackage{fancyhdr}
\usepackage{titletoc}
\usepackage{titlesec}
\usepackage{oldstyle}
\usepackage{lettrine}
%\usepackage{fontspec}


%%% Fonts
%\usepackage{gfsdidot}

%\defaultfontfeatures{Ligatures=TeX}

%\setmainfont[
%]{Bembo Std}

%\setmainfont[
%    BoldFont={Palatino Bold},
%    BoldItalicFont={Palatino Bold Italic},
%    SmallCapsFont={Aldus Small Caps & Oldstyle Fig}
%]{Aldus}
%\linespread{1.069}

%\setsansfont{Gill Sans}


\usepackage{mathpazo}\linespread{1.069}
\usepackage{avant}
%\renewcommand{\ttdefault}{cmtl} %% --- CmBright Typewriter Font

%%% Colors
\IfDefined{definecolor}{%

% Farbe der Ueberschriften
%\definecolor{sectioncolor}{RGB}{0, 51, 153} % Blau
%\definecolor{sectioncolor}{RGB}{0, 25, 152} % Blau (dunkler))
\definecolor{sectioncolor}{RGB}{0, 0, 0}     % Schwarz
%
% Farbe des Textes
\definecolor{textcolor}{RGB}{0, 0, 0}        % Schwarz
%
% Farbe fuer grau hinterlegte Boxen (fuer Paket framed.sty)
\definecolor{shadecolor}{gray}{0.90}

% Farben fuer die Links im PDF
\definecolor{pdfurlcolor}{rgb}{0,0,0.6}
\definecolor{pdffilecolor}{rgb}{0.7,0,0}
\definecolor{pdflinkcolor}{rgb}{0,0,0.0}
\definecolor{pdfcitecolor}{rgb}{0,0,0.6}

%% PDF-Linkfarben auf schwarz für den Druck:
% \definecolor{pdfurlcolor}{rgb}{0,0,0}
% \definecolor{pdffilecolor}{rgb}{0,0,0}
% \definecolor{pdflinkcolor}{rgb}{0,0,0}
% \definecolor{pdfcitecolor}{rgb}{0,0,0}

% The "ocre" color
\definecolor{orange}{RGB}{243,102,25}

% The "ocre" color
\definecolor{chaptergrey}{rgb}{0.6,0.6,0.6}

% Farben fuer Listings
\colorlet{stringcolor}{green!40!black!100}
\colorlet{commencolor}{blue!0!black!100}

} % Endif

%%% Styling of chapters, sections, subsections ...
\IfChapterDefined{

\titleformat{\chapter}[block]
{\normalfont\Huge\itshape}
%{\filleft\normalfont\fontsize{100}{130}\sffamily\bfseries\textcolor{chaptergrey}{\thechapter}}
{}
{0ex}
{}
[]

}

\renewcommand*{\othersectionlevelsformat}[3]{%
\makebox[0pt][r]{\textcolor{orange}{\bfseries#3}\hspace{0.5em}}}

\setkomafont{sectioning}{\rmfamily\itshape}
\setkomafont{descriptionlabel}{\rmfamily\itshape}

\setkomafont{section}{\Large}
\titlespacing*{\section}{0pt}{4ex plus 1ex minus .4ex}{1ex plus .2ex}
% * kills parindent 

\setkomafont{subsection}{\fontsize{12}{16}\selectfont}
\titlespacing*{\subsection}{0pt}{3ex plus .1ex minus .4ex}{.5ex plus .2ex}

\setkomafont{subsubsection}{\fontsize{11}{16}\selectfont}
\titlespacing*{\subsubsection}{0pt}{2ex plus .1ex minus .2ex}{.2ex plus .2ex}

%\renewcommand\paragraph{\@startsection{paragraph}{4}{\z@}%
%                                    {-2ex \@plus-.2ex \@minus .2ex}%
%                                    {0.1ex}%
%                                    {\normalfont\fontsize{11}{13}\sffamily\bfseries}}
%\renewcommand\subparagraph{\@startsection{subparagraph}{5}{\z@}%
%                                       {-2ex \@plus-.1ex \@minus .2ex}%
%                                       {0.1ex}%
%                                      {\normalfont\normalsize\sffamily\bfseries}}


%%% Table of contents
\setcounter{tocdepth}{1}
\setcounter{secnumdepth}{2}

\contentsmargin{0pc}
\IfChapterDefined{

\titlecontents{chapter}
  [0pc]% distance from left margin
  {\addvspace{1pc}\itshape}% global formatting before entry
  {\Large\thecontentslabel\enspace}% before entry when it has a number
  {\Large}% before entry when it does not have a number
  {\quad\thecontentspage}% the page number
  [\addvspace{0.1pt}]% after the entry

}

\titlecontents{section}
  [1.2pc]
  {\itshape}
  {\thecontentslabel\enspace}
  {}
  {\ \small\thecontentspage}


% Fancyhdr stuff
\newlength\myfancyhfoffset\setlength\myfancyhfoffset{0pt}
\pagestyle{fancy}
\fancyheadoffset{\myfancyhfoffset}
\fancyfootoffset{\myfancyhfoffset}
\IfChapterDefined{
\renewcommand{\chaptermark}[1]{\markboth{\normalfont\normalsize\scshape#1}{\normalfont\normalsize\scshape#1}}
}
%\renewcommand{\sectionmark}[1]{\markright{\normalfont\normalsize\thesection\hspace{5pt}\scshape#1}{}}
\renewcommand{\sectionmark}[1]{\markright{\normalfont\normalsize\scshape#1}}
\fancyhf{} \fancyhead[LE,RO]{\normalfont\normalsize\thepage}
\fancyhead[LO]{\itshape{\rightmark}}
\fancyhead[RE]{\itshape{\leftmark}}
\renewcommand{\headrulewidth}{0pt}
\fancypagestyle{plain}{
\fancyhead{}
\fancyfoot[C]{\normalfont\normalsize\thepage}
\renewcommand{\headrulewidth}{0pt}}

% Bibliography
\usepackage[%
	square,	 % for square brackets;
	comma,	 % to use commas as separaters;
	numbers, % for numerical citations;
	sort,    % orders multiple citations into the sequence in which they appear in the list of references;
]{natbib}

\bibliographystyle{alphaurl}    % labels are formed by author and year

%%% Listings
\lstset{
    basicstyle=\small\ttfamily,
    numbers=left,
    numberstyle=\tiny,
    stepnumber=2,
    numbersep=5pt,
    tabsize=2,
    extendedchars=true,
    breaklines=true,
    keywordstyle=[1]\textbf,
    keywordstyle=[2]\textbf,
    keywordstyle=[3]\textbf,
    keywordstyle=[4]\textbf,
    stringstyle=\color{stringcolor},
    showspaces=false,
    showtabs=false,
    showstringspaces=false,
    columns=fullflexible,
    frame=shadowbox,
    rulesepcolor=\color{black},
    framexleftmargin=4mm,
    xleftmargin=5mm,
    xrightmargin=1mm,
    numbers=left,
    stepnumber=1,
    numberblanklines=true,
    captionpos=t
 }
 \lstloadlanguages{% Check Dokumentation for further languages ...
         Java,
         Python
         %Pascal
         %C
         %C++
         %XML
         %HTML
 }
\lstdefinestyle{simple}{
    frame=none,
    framexleftmargin=1mm,
    xleftmargin=0mm,
    numbers=none
}
\lstdefinestyle{nonumbers}{
    framexleftmargin=1mm,
    xleftmargin=0mm,
    numbers=none
}

% Page Layout
\usepackage[top=3cm,left=3cm,right=3cm,bottom=4cm,a4paper]{geometry}

\IfPackageLoaded{typearea}{
   \IfPackageNotLoaded{geometry}{
      \typearea[current]{last}
   }
}

\raggedbottom % Allow variable page heights, probably bad for towside

%%% PDF & Url stuff
\hypersetup{
   % Farben fuer die Links
   pdfpagelabels,
   colorlinks=true,         % Links erhalten Farben statt Kaeten
   urlcolor=pdfurlcolor,    % \href{...}{...} external (URL)
   filecolor=pdffilecolor,  % \href{...} local file
   linkcolor=pdflinkcolor,  %\ref{...} and \pageref{...}
   citecolor=pdfcitecolor,  %
   % Links
   raiselinks=true,			 % calculate real height of the link
   breaklinks,              % Links berstehen Zeilenumbruch
   %backref=page,            % Backlinks im Literaturverzeichnis (section, slide, page, none)
   %pagebackref=true,        % Backlinks im Literaturverzeichnis mit Seitenangabe
   verbose,
   hyperindex=true,         % backlinkex index
   linktocpage=true,        % Inhaltsverzeichnis verlinkt Seiten
   hyperfootnotes=false,     % Keine Links auf Fussnoten
   % Bookmarks
   bookmarks=true,          % Erzeugung von Bookmarks fuer PDF-Viewer
   bookmarksopenlevel=1,    % Gliederungstiefe der Bookmarks
   bookmarksopen=true,      % Expandierte Untermenues in Bookmarks
   bookmarksnumbered=true,  % Nummerierung der Bookmarks
   bookmarkstype=toc,       % Art der Verzeichnisses
   % Anchors
   plainpages=false,        % Anchors even on plain pages ?
   pageanchor=true,         % Pages are linkable
   % PDF Informationen
   pdftitle={},             % Titel
   pdfauthor={},            % Autor
   pdfcreator={LaTeX, hyperref, KOMA-Script}, % Ersteller
   %pdfproducer={pdfeTeX 1.10b-2.1} %Produzent
   pdfdisplaydoctitle=true, % Dokumententitel statt Dateiname im Fenstertitel
   pdfstartview=FitH,       % Dokument wird Fit Width geaefnet
   pdfpagemode=UseOutlines, % Bookmarks im Viewer anzeigen
   pdfpagelabels=true,           % set PDF page labels
   %pdfpagelayout=TwoPageRight, % zweiseitige Darstellung: ungerade Seiten
   									 % rechts im PDF-Viewer
   pdfpagelayout=SinglePage, % einseitige Darstellung
}

% URLs
\IfDefined{urlstyle}{
	\urlstyle{tt} %sf
}



%%% Footnoes
% Keine hochgestellten Ziffern in der Fussnote (KOMA-Script-spezifisch):
\deffootnote{1.5em}{1em}{\makebox[1.5em][l]{\thefootnotemark}}
\addtolength{\skip\footins}{\baselineskip} % Abstand Text <-> Fussnote

\setlength{\dimen\footins}{10\baselineskip} % Beschraenkt den Platz von Fussnoten auf 10 Zeilen

\interfootnotelinepenalty=10000 % Verhindert das Fortsetzen von
                                % Fussnoten auf der gegenüberligenden Seite

%%% Figures and stuff
\usepackage{caption}
% Aussehen der Captions
\captionsetup{
   margin = 20pt,
   font = {small},
   labelfont = {rm},
   indention = 0em,  % Einruecken der Beschriftung
   labelsep = endash, %period, space, quad, newline
   justification = RaggedRight, % justified, centering
   singlelinecheck = true, % false (true=bei einer Zeile immer zentrieren)
   position = bottom %top
}
%%% Bugfix Workaround
\DeclareCaptionOption{parskip}[]{}
\DeclareCaptionOption{parindent}[]{}

% Aussehen der Captions fuer subfigures (subfig-Paket)
\IfPackageLoaded{subfig}{
 \captionsetup[subfloat]{%
   margin = 10pt,
   font = {small,rm},
   labelfont = {small,bf},
   format = plain, % oder 'hang'
   indention = 0em,  % Einruecken der Beschriftung
   labelsep = space, %period, space, quad, newline
   justification = RaggedRight, % justified, centering
   singlelinecheck = true, % false (true=bei einer Zeile immer zentrieren)
   position = bottom, %top
   labelformat = parens % simple, empty % Wie die Bezeichnung gesetzt wird
 }
}

% Aendern der Bezeichnung fuer Abbildung und Tabelle
% \addto\captionsngerman{% "captionsgerman" fuer alte  Rechschreibung
%   \renewcommand{\figurename}{Abb.}%
%   \renewcommand{\tablename}{Tab.}%
% }

% Inhalte von List of Figures
\IfPackageLoaded{subfig}{
	\setcounter{lofdepth}{1}  %1 = nur figures, 2 = figures + subfigures
}
